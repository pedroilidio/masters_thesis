%% USPSC-Agradecimentos.tex
\begin{agradecimentos}
À Universidade de São Paulo e ao Instituto de Física de São Carlos (IFSC), por proverem os meios e os ambientes pelos quais este estudo pôde se desenvolver.

À Neusa, do Serviço de Biblioteca e Informação do IFSC, pelo atencioso auxílio com a normalização do presente documento.

Aos estimados orientadores Prof. Dr. Otavio Henrique Thiemann e Prof. Dr. Ricardo Cerri, pelas numerosas ideias, inspiração e direcionamento. Pelo acolhimento frente às repentinas readaptações acadêmicas que se fizeram necessárias.
Pela confiança e compreensão com os atrasos, com os afluentes e com os contratempos durante todo o processo.

Ao André Alves, pelas várias discussões e colaboraç\~oes enriquecedoras das quais grande parte desse trabalho teve origem.

Ao Prof. Dr. Ricardo de Marco, em memória, pelos muitos ensinamentos nos poucos anos que pudemos conviver.

Aos irmãos acadêmicos João Paulo, Luíza e Renan, pelo amparo e companhia essenciais durante os percalços do luto, da pandemia e das incertezas da vida.

Aos profissionais da saúde mental Marina e Pedro Paulo, pelo imprescindível apoio.

Aos prezados coabitantes Kauê e Roberto, pelo convívio agradável e encorajamento diário.
Por ouvirem sempre gentilmente os desabafos e aflições.

Às demais prezadas ``Batatas'': Beatriz, Estevão, Gabriela e Nathan, pelas aconchegantes memórias, pela paciência com as ausências e pelo carinho invariável.

Às famílias de Minas Gerais e de São Paulo, pela sempre amorosa e calorosa torcida.

Aos meus irmãos André e Caio, pela amizade perene e inabalável. Por compartilharem comigo seu crescimento e serem parte fundamental do meu.

%Aos meus pais Marisa e Sandro, pelo apoio e compreensão incondicionais.
%Por me permitirem descobrir quem eu sou
%
%Os meus próprios caminhos
%Deixar desabrocharem meus próprios caminhos
%Selvagens

%Por cultivarem despodados os meus próprios caminhos, por me 
%
%Por cultivarem amorosamente selvagens e
%despodados os meus próprios caminhos.
%
%Deixar crescer 
%Regarem com água potável 
%
%Por me apoiarem e me levantarem e me amparem mesmo depois das decisões erradas

Aos inestimáveis pais Marisa e Sandro, por cultivarem despodados meus caminhos, me permitindo errar, desviar, retornar, e descobrir quem eu sou a cada passo. Pelo amor incondicional com que me recebem após cada dificuldade.
% Aos meus inestimáveis pais Marisa e Sandro, por cultivarem despodados meus caminhos, por me permitirem errar, desviar, retornar, e descobrir quem eu sou a cada passo. Pelo amor incondicional com que me recebem após cada tropeço.

%À querida Alê, por não desistir de mim nem nas piores tempestades.
%Por fazer a aurora 
À querida Alê, por trazer a aurora mesmo às piores tempestades. $\tau$.

%A todos os esforços para a democratização da pesquisa científica.
%To all the efforts for the democratization of scientific research.
%To all the efforts for the democratization of scientific research.

\vfill

This study was financed in part by the Coordenação de Aperfeiçoamento de Pessoal de Nível Superior - Brasil (CAPES) - Finance Code 001.

\end{agradecimentos}
% Àqueles a quem a ciência não é gentil, mas ainda assim seu objeto de amor
% Àqueles a quem o processo não é gentil, mas não deixam de ter a ciência como objeto de amor
% Àqueles a quem o processo não é gentil, mas mantêm a ciência como objeto de amor.
% 
% To those to whom the process is not kind, but still have science as their object of love
% ---
%To all of those who persevere through harsh conditions to make science happen.
%To all of those who must endure to make science happen.
%To all of those who specially endure to satisfy their desire of partaking in scientific creation.
%To those who specially endure to partake in scientific creation.
%To those who specially persevere to partake in scientific creation.