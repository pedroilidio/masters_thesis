% Define struct-like syntax for objects
\SetKwRepeat{Node}{NodeObject \{}{\}}
\SetKwRepeat{Split}{SplitObject \{}{\}}

%keywords for local models
\SetKwFunction{FindSplit}{FindSplit}
\SetKwFunction{KwCombine}{combine}
\SetKwFunction{KwPrototype}{prototype}
\SetKwData{primaryRows}{primaryRowsEstimator}
\SetKwData{primaryColumns}{primaryColumnsEstimator}
\SetKwData{secondaryRows}{secondaryRowsEstimator}
\SetKwData{secondaryColumns}{secondaryColumnsEstimator}
\SetKwFunction{train}{train}
\SetKwFunction{predict}{predict}


% \newcommand{\algFindBestSplit}{
% \begin{function}
% \SetKwFunction{SplitQuality}{SplitQuality}
% \SetKwFunction{BestSplit}{BestSplit}
% \SetKwFunction{SplitDataset}{SplitDataset}
% %\tcp{Index of the feature ($X$ column) used in the best split found}
% Initialize \BestSplit object\;
% \BlankLine
% 
% \ForAll {$j \in \mathbb{N},\;1 \le j \le n_c$}{
%     Sort $X_a$ rows according to $X_a\el{:, j}$\;
%     Sort $Y$ rows according to $X_a\el{:, j}$\;
%     \BlankLine
%     \ForAll {$i \in \mathbb{N},\;1 \le i \le n_r$}{
%         $X_l, X_r, Y_l, Y_r \gets$ \SplitDataset{$X_a$, $Y$, $i$}\;
%         % TODO: organize impurity funcs.
%         $Q\gets$ \SplitQuality{$X_l$, $X_r$, $Y_l$, $Y_r$}\;
%         \BlankLine
%         \If{$Q >$\BestSplit.Quality}{
%             \BestSplit $\gets$ \Split{}{
%                 $X_l$, $X_r$, $Y_l$, $Y_r$\;
%                 Quality $\gets Q$\; 
%                 Index $\gets i$\; 
%                 Feature $\gets j$\; 
%                 Threshold $\gets \frac{1}{2}(X_a\el{i, j} + X_a\el{i+1, j})$\;
%             }
%         }
%     }
%     \BlankLine
%     \Return \BestSplit
% }
% \caption{FindBestSplit($X_a$, $Y$)}
% \label{alg:find_best_split}
% \end{function}}

\newcommand{\algFindBestSplit}{
\begin{function}[htb]
    \KwIn{A partition of the training data in a given node.}
    \KwOut{The highest quality score $Q^*$ found among all splits evaluated,
        with its corresponding feature column $f^*$ and threshold value $t^*$.}
    \BlankLine
    %\SetKwFunction{SplitQuality}{SplitQuality}
    %\SetKwFunction{BestSplit}{BestSplit}
    %\SetKwFunction{SplitDataset}{SplitDataset}
    %\tcp{Index of the feature ($X$ column) used in the best split found}
    %Initialize \BestSplit object\;
    %Calculate $\savg{Y\el{i}}_i$ and $\savg{(Y\el{i})^2}_i$\;
    Initialize $S_r$ and $S_l$ as a $|Y|_j$-sized vectors\;%\tcp{Will store sums of $Y_r$ and $Y_l$ columns}
    \BlankLine
    %$S_l \gets 0$ \tcp*{The same but for $Y_l$}
    % $f^* \gets 0$ \tcp*{best split's feature}
    % $Q^* \gets 0$ \tcp*{best split's quality}
    % $t^* \gets 0$ \tcp*{best split's threshold}
    $Q^*$, $f^*$, $t^* \gets \mathbf {0}$\;
    \BlankLine
    Draw $\tilde n_f$ columns (features) of $X$ without replacement\;
    \BlankLine
    
    \ForEach {feature index $f$ of the $\tilde n_f$ drawn features}{
        $n_l \gets 0$\tcp*{Holds $|Y_l|_i$}
        %\tcp{$S_r$ (resp. $S_l$) will store sums for each $Y_r$ ($Y_l$) column}
        $S_r \gets \sum_i Y\el{i\jhat}$\tcp*{Holds $\sum_i Y_r\el{i\jhat}$} \label{algline:gmo_loop1}
        $S_l \gets \mathbf{0}$\tcp*{Holds $\sum_i Y_l\el{i\jhat}$}
        %\ForEach {column index $\jhat$ of $Y$}{
        %    $S_r\el \jhat \gets \sum_i Y\el{i\jhat}$\;
        %    $S_l\el \jhat \gets 0$\;
        %}
        \BlankLine
        Get the permutation $P$ that sorts $X\el{\cdot f}$\;
        %Apply $p$ to $Y$ and $X$, yielding $Y_\text{permuted}$ and $X_\text{permuted}$\;
        %Calculate $Y_\text{permuted}$ and $X_\text{permuted}$ by applying $p$ to $Y$'s and $X$'s rows\;
        %$Y_\text{permuted}$, $X_\text{permuted}\gets$ Apply $p$ to $Y$ and $X$\;
        \BlankLine
        Apply $P$ to $Y$'s and $X\el{\cdot f}$'s rows:\\
        \Indp $Y_P$, $X_P\gets P(Y)$, $P(X\el{\cdot f})$\;\Indm
        \BlankLine
        
        % TODO: use \ihat for loop variable
        % TODO: set inner loop as another function to include random splits?
        \ForEach {row index $\ihat$ of $Y_P$}{
            $n_l \gets n_l+1$\;
            \ForEach {column index $\jhat$ of $Y_P$}{ \label{algline:gmo_loop2}
                $S_r\el \jhat \gets S_r\el \jhat - Y_P\el{\ihat\jhat}$\;
                $S_l\el \jhat \gets S_l\el \jhat + Y_P\el{\ihat\jhat}$\;
            }
            %$\savg{(\savg{Y_l\el{ij}}_i\el j)^2} \gets S_l / n_l$\;
            %$\savg{Y_r\el{ij}}\el j \gets S_r / (|Y|_i - n_l)$\;
            \BlankLine
            % Calculate $Q$ with Eq. \ref{eq:quality}\;
            Use $S_l$, $S_r$ and $n_l$ to calculate $Q$ (Eq. \ref{eq:q_optimization}).
            Notice that other node-specific constants might be needed\;
            %Calculate $Q$ with Eq. \ref{eq:q_optimization}\;
            \BlankLine
            \If{$Q > Q^*$}{
                $Q^* \gets Q$\; 
                $f^*\gets f$\; 
                $t^* \gets \frac{1}{2}(X_P\el{\ihat} + X_P\el{\ihat+1})$\;
            }
        }
    }
    \BlankLine
    \Return $Q^*$, $f^*$, $t^*$\;
    \caption{FindSplitBest($X$, $Y$)}
    \label{alg:find_best_split}
\end{function}}


\newcommand{\algFindRandomSplit}{
\begin{function}[htb]
    \KwIn{A partition of the training data in a given node.}
    \KwOut{The highest quality score $Q^*$ found among all splits evaluated,
        with its corresponding feature column $f^*$ and threshold value $t^*$.}
    \BlankLine
    \SetKwFunction{min}{min}
    \SetKwFunction{max}{max}
    %$S_l \gets 0$ \tcp*{The same but for $Y_l$}
    % $f^* \gets 0$ \tcp*{best split's feature}
    % $Q^* \gets 0$ \tcp*{best split's quality}
    % $t^* \gets 0$ \tcp*{best split's threshold}
    $Q^*$, $f^*$, $t^* \gets \mathbf {0}$\;
    \BlankLine
    Draw $\tilde n_f$ columns (features) of $X$ without replacement\;
    \BlankLine
    
    \ForEach {feature index $f$ of the $\tilde n_f$ drawn features}{
        Find \min{$X\el{\cdot f}$} and \max{$X\el{\cdot f}$}\;
        Draw a random threshold value $t\in \mathbb{R}$ so that
            \min{$X\el{\cdot f}$}$ < t < $\max{$X\el{\cdot f}$}\;
        \BlankLine
        % Calculate $Q$ with Eq. \ref{eq:quality}\;
        Calculate $Q$ for the drawn $t$ (Eq. \ref{eq:q_optimization}) \tcp*{$O(|Y|)$}
        
        \BlankLine
        \If{$Q > Q^*$}{
            $Q^* \gets Q$\; 
            $f^*\gets f$\; 
            $t^* \gets \frac{1}{2}(X_P\el{\ihat} + X_P\el{\ihat+1})$\;
        }
    }
    \BlankLine
    \Return $Q^*$, $f^*$, $t^*$
    \label{alg:find_random_split}
    \caption{FindSplit$_\text{random}$($X$, $Y$)}
\end{function}}


\newcommand{\algBuildTree}{
\begin{function}[htb]
    \SetKwFunction{BuildTree}{BuildTree}
    \SetKwFunction{DecideToStop}{DecideToStop}
    \KwIn{The training data for the current node.}
    \KwOut{Current node, with all information of subsequent splits.}
    
    $Q^*$, $f^*$, $t^*\gets$ \FindSplit($X$, $Y$, $\tilde n_f$)\;
    \BlankLine
    \tcp{Many stopping criteria are possible}  % refer to discussion
    \eIf{\DecideToStop{$Q^*$, $f^*$, $t^*$, $X$, $Y$}}{
        \Return \Node {}{
            isLeaf $\gets$ True\\
            $X_\text{leaf}\gets X$\\
            $Y_\text{leaf}\gets Y$\\
        }
    }{
        Get $X_l$, $Y_l$, $X_r$, $Y_r$ from $f^*$ and $t^*$ (Eq. \ref{eq:datasplit})\;
        \BlankLine
        \Return \Node {}{
            isLeaf $\gets$ False\\
            childLeft $\gets$ \BuildTree{$X_l$, $Y_l$}\\
            childRight $\gets$ \BuildTree{$X_r$, $Y_r$}\\
            feature $\gets$ $f^*$\\
            threshold $\gets$ $t^*$\\
        }
    }
    \caption{BuildTree($X$, $Y$): Recursively build a Decision Tree}%, $\tilde n_f$)}
    \label{alg:buildtree}
\end{function}}


\newcommand{\algPredict}{
\begin{function}[htb]
\KwIn{A new interaction sample to be evaluated and the root node of a Decision Tree.}
\KwOut{The Decision Tree's predicted value for the given sample attributes.}

Node $\gets$ RootNode\;
\BlankLine
\While{Node is not a leaf}{
    \eIf{$x\el{\text{Node.feature}}>$ Node.threshold}{
        Node $\gets$ Node.childRight
    }{
        Node $\gets$ Node.childLeft
    }
}
\BlankLine
\Return \KwPrototype{Node.X, Node.Y}\label{ln:prototype}
\caption{Predict(RootNode, $x$): Compute a Decision Tree's prediction.}
\label{alg:predict}
\end{function}}


\newcommand{\algFindBipartiteSplit}{
\begin{function}[htb]
    \KwIn{A partition of the bipartite training data in a given node.
        $X$ encodes one design matrix for each axis, $X_1$ and $X_2$.}
    \KwOut{The highest quality score $Q^*$ found among all splits evaluated
        in both row and column directions,
        with its corresponding feature column $f^*$ and threshold value $t^*$.}
    \SetKwData{adapter}{adapterStrategy}
    \BlankLine
    \eIf{\adapter is GSO (\autoref{sec:bgso_trees})}{
        \tcp{Build $Y$ proxies $\tilde Y_1$ and $\tilde Y_2$ (\autoref{eq:y_proxies})}
        $\tilde Y_1 \gets Y\el{\cdot}\mel j$\;
        $\tilde Y_2 \gets Y\mel i\el \cdot$\;
        %$\tilde Y_1\el i \gets (\sum_j Y\el{ij})\el i$\;
        %$\tilde Y_2 \el i \gets (\sum_i Y\el{ij})\el j$\;
    }
    {
        \tcp{Using GMO strategy, no proxies are used}
        $\tilde Y_1 \gets Y$\;
        $\tilde Y_2 \gets Y^\intercal$\;
    }
    \BlankLine
    %RowSplitIndex, RowSplitFeature $\gets$ \FindSplit{$X_1$, $\tilde Y_1$}\;
    %ColumnSplitIndex, ColumnSplitFeature $\gets$ \FindSplit{$X_2$, $\tilde Y_2$}\;
    \tcp{Generate a split in each axis. Get each split's position, feature and quality score}
    $Q_r^*, f_r^*, t_r^*\gets$ \FindSplit{$X_1$, $\tilde Y_1$}\;
    $Q_c^*, f_c^*, t_c^*\gets$ \FindSplit{$X_2$, $\tilde Y_2$}\;
    \BlankLine
    %Calculate ColumnSplitQuality and RowSplitQuality\;
    
    \eIf{$Q_r^* > Q_c^*$}{
        \Return $Q_r^*$, $f_r^*$, $t_r^*$
    }{
        \tcp{$f_c^*$ value lets clear its $X_2$ ownership}
        \Return $Q_c^*$, $f_c^*$, $t_c^*$
    }
    
    \caption{FindBipartiteSplit($X$, $Y$)}
    \label{alg:find_bipartite_split}
\end{function}}

\newcommand{\algSplitDataset}{
\begin{procedure}[htb]
$Y_l \gets Y\el{1..\ihat, 1..n_{s2}}$\;
$Y_r \gets Y\el{\hat i..n_{s1}, 1..n_{s2}}$\;
$X_l \gets X\el{1..\ihat,\;\cdots}$\;
$X_r \gets X\el{\hat i..n_{s1}, 1..n_{s2}}$\;
\Return $X_l$, $X_r$, $Y_l$, $Y_r$\;
\caption{SplitDataset($X$, $Y$, $i$)}
\end{procedure}
}


\newcommand{\algForest}{
\begin{procedure}[htb]

% TODO \tilde is already Y proxies
Build $\tilde n_s$ bootstrap samples $\tilde X$, $\tilde Y$ from the original data.

Grow 

\caption{Forest($X$, $Y$)}
\end{procedure}
}



\newcommand{\algTrainLocalModel}{
\begin{function}[htb]
    \KwIn{The bipartite training dataset.}
    \KwOut{A bipartite local model.}
    \BlankLine
    
    %Train multioutput primary rows estimator on $X_1$ and $Y$.\;
    %Train multioutput primary columns estimator on $X_2$ and $Y^\intercal$.\;
    \train{\primaryRows, $X_1$, $Y$}\;
    \train{\primaryColumns, $X_2$, $Y^\intercal$}\;
    \BlankLine
    
    \Return \primaryRows, \primaryColumns
    
    \caption{TrainLocalModel($X$, $Y$)}
    \label{alg:train_local_model}
\end{function}}


\newcommand{\algPredictLocalModel}{
\begin{function*}[htb]
    \KwIn{
        The trained primary models and the unseen sample matrices $X_\text{new}$ for both axes.
    }
    \KwOut{
        $Y_\text{pred}$ predictions for each interaction provided.
    }
    \BlankLine
    
    %\tcp{Use the primary rows estimator to predict new label rows from $X_1$}
    $Y_\text{new rows} \gets$ \predict{\primaryRows, $X_{1\text{new}}$}
    
    \BlankLine
    \BlankLine
    %\tcp{Use the primary columns estimator to predict new label columns from $X_2$}
    $Y_\text{new cols} \gets$ \predict{\primaryColumns, $X_{2\text{new}}$}
    \BlankLine
    \BlankLine
    
    \If{Secondary estimators consider label dependencies}{
        \tcp{Concatenate known rows and columns labels to the primary predictions}
        $Y_\text{new cols}
            \gets \begin{bmatrix}
                Y^\intercal \\ Y_\text{new cols}
            \end{bmatrix}$\;
        $Y_\text{new rows}
            \gets \begin{bmatrix}
                Y \\ Y_\text{new rows}
            \end{bmatrix}$\;
        \tcp{Otherwise, if label columns are considered independently, this step is not necessary}
    }
    \BlankLine
    \BlankLine
    %\tcp{Train the secondary columns estimator on $X_{2\text{new}}$ and $Y_\text{new rows}^\intercal$}
    \train{\secondaryRows, $X_1$, $Y_\text{new cols}^\intercal$}\;
    \train{\secondaryColumns, $X_2$, $Y_\text{new rows}^\intercal$}\;
    \BlankLine
    \BlankLine
    $Y_\text{pred rows}\gets$ \predict{\secondaryRows, $X_{1\text{new}}$}\;
    $Y_\text{pred cols}\gets$ \predict{\secondaryColumns, $X_{2\text{new}}$}\;
    %\tcp{Train the secondary rows estimator on $X_{1\text{new}}$ and $Y_\text{new cols}$}
    
    \BlankLine
    \BlankLine
    \If{Secondary estimators consider label dependencies}{
        \tcp{Skip predictions not referring to $X_{1\text{new}}$ and $X_{2\text{new}}$}
        $Y_\text{pred rows}\gets Y_\text{pred rows}\el{\cdot j> |X_1|_i}$\;
        $Y_\text{pred cols}\gets Y_\text{pred cols}\el{\cdot j> |X_2|_i}$\;
    }
    
    \BlankLine
    \BlankLine
    \Return \KwCombine{$Y_\text{pred rows}$, $Y_\text{pred cols}^\intercal$} \label{ln:combine_local_outputs}
    
    \caption{PredictLocalModel(primary models, $X_\text{new}$)}
    \label{alg:predict_local_model}
\end{function*}}


\newcommand{\algNRLMFTrain}{  % TODO adagrad
\begin{function}[tbh]
    \SetKwData{maxIter}{max\_iter}
    \KwIn{$Y$, the training labels matrix to be approximated.}
    \KwIn{$S_1$, $S_2$, Similarity matrices among instances of each axis.}
    \KwIn{$\alpha$, the positive importance factor.}
    \KwIn{$\lambda_1$, $\lambda_2$, quadratic regularization factors.}
    \KwIn{$\beta_1$, $\beta_2$, neighborhood regularization factors.}
    \KwIn{$\eta$, the learning rate.}
    %\KwIn{\maxIter, the maximum number of iterations.}
    \KwOut{$U$, $V$, the resulting latent feature matrices.}
    \BlankLine
    Optionally precompute constant factors of the gradient
    (Equations \ref{eq:nrlmf_gradient_U} and \ref{eq:nrlmf_gradient_V}),
    %(\cref{eq:nrlmf_gradient_U,eq:nrlmf_gradient_V}),  %TODO
    such as
    $(\lambda_2\mathbb{I} + \beta_2 L_2)$, $[(1-\alpha)Y-1]$ or $\alpha Y$\;
    \BlankLine
    Initialize $U$ and $V$ with normally-distributed random values\;
    \BlankLine
    $T_U, T_V \gets \mathbf{0}$\tcp*{Initialize gradient accumulators}

    \BlankLine
    % \Repeat{Maximum number of iterations or satisfactory gain is achieved}{
    \While{Stop conditions are not met}{
        \tcp{Update U}
        Obtain $G_U$ through \autoref{eq:nrlmf_gradient_U}\;
        $T_U \gets T_U + G_U^2$\;
        $U \gets U + \eta \frac{G_U}{\sqrt{T_U}}$\;
        \BlankLine
        \tcp{Update V}
        Obtain $G_V$ through \autoref{eq:nrlmf_gradient_V}\;
        $T_V \gets T_V + G_V^2$\;
        $V \gets V + \eta \frac{G_V}{\sqrt{T_V}}$\;
    }
    \BlankLine
    \Return $U$, $V$
    \caption{TrainNRLMF($Y$, $S_1$, $S_2$, $\alpha$, $\lambda_1$, $\lambda_2$, $\beta_1$, $\beta_2$, $\eta$): Train an NRLMF model.}
    \label{alg:nrlmf_train}
\end{function}}